%\documentclass[wcp,gray]{jmlr} % test grayscale version
 %\documentclass[wcp]{jmlr}% former name JMLR W\&CP
\documentclass[pmlr]{jmlr}% new name PMLR (Proceedings of Machine Learning)

 % The following packages will be automatically loaded:
 % amsmath, amssymb, natbib, graphicx, url, algorithm2e

 %\usepackage{rotating}% for sideways figures and tables
\usepackage{longtable}% for long tables

 % The booktabs package is used by this sample document
 % (it provides \toprule, \midrule and \bottomrule).
 % Remove the next line if you don't require it.
\usepackage{booktabs}
 % The siunitx package is used by this sample document
 % to align numbers in a column by their decimal point.
 % Remove the next line if you don't require it.
\usepackage[load-configurations=version-1]{siunitx} % newer version
 %\usepackage{siunitx}

\makeatletter
\def\set@curr@file#1{\def\@curr@file{#1}} %temp workaround for 2019 latex release
\makeatother

 % The following command is just for this sample document:
\newcommand{\cs}[1]{\texttt{\char`\\#1}}

 % Define an unnumbered theorem just for this sample document:
\theorembodyfont{\upshape}
\theoremheaderfont{\scshape}
\theorempostheader{:}
\theoremsep{\newline}
\newtheorem*{note}{Note}

 % change the arguments, as appropriate, in the following:
\jmlrvolume{149}
\jmlryear{2021}
\jmlrworkshop{Machine Learning for Healthcare}

% Short headings should be running head and authors last names
% \ShortHeadings{A Really Awesome MLHC Article}{Lastname, PhD and Lastname, MD}
% \firstpageno{1}

\title[Short Title]{Contrastive Learning of Electrocardiogram signals for the inference of cardiac pressure}

\author{\Name{Hyewon Jeong}
       \Email{hyewonj@mit.edu}\\
       \addr Computer Science and Artificial Intelligence Laboratory\\
       Massachusetts Institute of Technology\\
       Cambridge, MA, United States of America
       \AND
       \Name{Marzyeh Ghassemi}
       \Email{mghassem@mit.edu}\\
       \addr Computer Science and Artificial Intelligence Laboratory\\
       Massachusetts Institute of Technology\\
       Cambridge, MA, United States of America
       \AND
       \Name{Collin Stultz}
       \Email{name@email.edu}\\
              \addr Computer Science and Artificial Intelligence Laboratory\\
       Massachusetts Institute of Technology\\
       dCambridge, MA, United States of America
       }

\editor{Editor's name}

\begin{document}

\maketitle

\begin{abstract}
Abnormal cardiac signals in an electrocardiogram (ECG) are used to screen the heart condition.
Physicians oftentimes order diagnostic tests which include invasive procedures.
For instance, the function of the left heart can be precisely measured by the pulmonary capillary wedge pressure (PCWP) which is invaisve.
To reduce the risk of such invasive procedures, we can build a model to infer PCWP using non-invasive ECG screening.
Moreover, we can leverage the vast amount of unlabeled ECG dataset to learn a useful representation. %why contrastive learning?
Yet, an important question remains; what are the good views (\cite{tian2020makes}) of ECGs for contrastive learning?
Previous works have tried to include ECGs from the same patients (\cite{diamant2021patient}), different augmentations from the same ECG (\cite{gopal20213kg}), and different projections of the same ECG (\cite{kiyasseh2021clocs}) to define proper `views' for contrastive learning. However, none of them has tried to explore the various views from different ECGs and compare those. In this project, we define different views according to \textit{age, sex, racial groups} with and without augmentation (scaling, adding Gaussian noise, shifting, augmentation on spectrogram domain (\cite{park2019specaugment}). Then compare the learned representation over the downstream task - classification and regression of PCWP.
%need to write why this is important
\end{abstract}

%ECG of a patient with left heart failure might have a sign of left heart hypertrophy in his/her ECG signal.
% Why contrastive learning?
% How's view defined?




%\input{Sections/1_Introduction}
%\input{Sections/2_Approach}
%\input{Sections/3_Methods}
%
%\section{Cohort}
%
%\emph{This section is optional, more theoretical work may not need
%  this section.  However, if you are using health data, then you need
%  to describe it carefully so that the clinicians can validate the
%  soundness of your choices.}
%
%Describe the cohort.  Give us the details of any inclusion/exclusion
%criteria, what data were extracted, how features were processed,
%etc.  Recommended headings include:
%
%\subsection{Cohort Selection}
%with choice of criteria and basic numbers, as well as any relevant
%information about the study design (such how cases and controls were
%identified, if appropriate),
%
%\subsection{Data Extraction}
%with what raw information you extracted or collected, including any
%assumptions and imputation that may have been used, and
%
%\subsection{Feature Choices}
%with how you might have converted the raw data into features that were
%used in your algorithm.
%
%Cohort descriptions are just as important as methods details and code
%to replicate a study.  For more clinical application papers, each of
%the sections above might be several paragraphs because we really want
%to understand the setting.
%
%For the submission, please do \emph{not} include the name of the
%institutions for any private data sources.  However, in the
%camera-ready, you may include identifying information about the
%institution as well as should include any relevant IRB approval
%statements.
%
%\input{Sections/4_Results}
%
%\paragraph{Limitations}
%
%Explain when your approach may not apply, or things you could not
%check.  \emph{Discussing limitations is essential.  Both ACs and
%  reviewers have been coached to be skeptical of any work that does
%  not consider limitations.}

% ACKNOWLEDGEMENTS ONLY GO IN THE CAMERA-READY, NOT THE SUBMISSION
% \acks{Many thanks to all collaborators and funders!}

\bibliography{reference}

\appendix
\section*{Appendix A.}

Some more details about those methods, so we can actually reproduce
them.  After the blind review period, you could link to a repository
for the code also.  \emph{MLHC values both rigorous evaluation as well
  as reproduciblity.}

\end{document}
