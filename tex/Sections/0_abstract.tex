\begin{abstract}
Abnormal cardiac signals in an electrocardiogram (ECG) are used to screen the heart condition.
Physicians oftentimes order diagnostic tests which include invasive procedures.
For instance, the function of the left heart can be precisely measured by the pulmonary capillary wedge pressure (PCWP) which is invaisve.
To reduce the risk of such invasive procedures, we can build a model to infer PCWP using non-invasive ECG screening.
Moreover, we can leverage the vast amount of unlabeled ECG dataset to learn a useful representation. %why contrastive learning?
Yet, an important question remains; what are the good views (\cite{tian2020makes}) of ECGs for contrastive learning?
Previous works have tried to include ECGs from the same patients (\cite{diamant2021patient}), different augmentations from the same ECG (\cite{gopal20213kg}), and different projections of the same ECG (\cite{kiyasseh2021clocs}) to define proper `views' for contrastive learning. However, none of them has tried to explore the various views from different ECGs and compare those. In this project, we define different views according to \textit{age, sex, racial groups} with and without augmentation (scaling, adding Gaussian noise, shifting, augmentation on spectrogram domain (\cite{park2019specaugment}). Then compare the learned representation over the downstream task - classification and regression of PCWP.
%need to write why this is important
\end{abstract}

%ECG of a patient with left heart failure might have a sign of left heart hypertrophy in his/her ECG signal.
% Why contrastive learning?
% How's view defined?

